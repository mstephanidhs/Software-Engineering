\documentclass{article}

\makeatletter
\renewcommand{\fnum@figure}{Εικόνα \thefigure}
\makeatother

\usepackage[greek, english]{babel}
\usepackage{alphabeta}
\usepackage{atbegshi, picture}

% Set page size and margins
% Replace `letterpaper' with`a4paper' for UK/EU standard size
\usepackage[letterpaper,top=2cm,bottom=2cm,left=3cm,right=3cm,marginparwidth=1.75cm]{geometry}

% Useful packages
\usepackage{amsmath}
\usepackage{graphicx}
\usepackage[colorlinks=true, allcolors=blue]{hyperref}
\usepackage[utf8]{inputenc}
\usepackage{indentfirst}

\addto\captionsenglish{
  \renewcommand{\contentsname}
    {Περιεχόμενα}
}

% \title{Feasibility Study}
% \date{}

\begin{document}
% \maketitle

\begin{titlepage}
   \begin{center}
       \vspace*{1cm}

       \textbf{\huge Feasibility Study}

       \vspace{0.5cm}
        Τεχνολογία Λογισμικού
            
       \vspace{1cm}

       \textbf{Κατερίνα Μητροπούλου}
       
       \begin{figure}[!htb]
        \centering
        \includegraphics[width=0.5\textwidth]{logo.png}
        \end{figure}
        
        \vspace{0.5cm}
        
        \begin{figure}[!htb]
        \centering
        \includegraphics[width=0.5\textwidth]{ceid.jpg}
        \end{figure}


       \vfill
            
       Τεχνικό Κείμενο για την Τεχνολογία Λογισμικού\\
            
       \vspace{0.5cm}
            
       CEID, ECE\\
       University of Patras\\
            
   \end{center}
\end{titlepage}



\noindent Η ομάδα μας

\begin{enumerate}
  \item Βεργίνης Δημήτριος, ΑΜ: 10166634 , ECE
  \item Βλαχογιάννης Δημήτριος, ΑΜ: 1067371, CEID
  \item Κούρου Αγγελική, ΑΜ: 1067499 , CEID
  \item Μητροπούλου Αικατερίνα - Quality Manager, ΑΜ: 1067409, CEID
  \item Στεφανίδης Μάριος - Project Manager, ΑΜ:1067458, CEID
\end{enumerate}

{
  \hypersetup{linkcolor=black}
  \tableofcontents
}

\section{Εισαγωγή}

\subsection{Οικονομική Σκοπιμότητα}

Από τη στιγμή που πρόκειται για μία εφαρμογή που αποθηκεύει δεδομένα (για τους αθενείς και το ιατρικό προσωπικό) είναι απαραίτητη η χρήση ενός εξυπηρετητή (server).Η αποθήκευση όμως των δεδομένων θα γίνεται στους server του εκάστοτε νοσοκομείου, επομένως δεν απαιτείται η ενοικίαση ή η αγορά κάποιου server από την ομάδα σχεδίασης της εφαρμογής. \newline \par
Για να έχουν πρόσβαση τα νοσοκομειακά ιδρύματα στην εφαρμογή θα χρειαστεί να την αγοράσουν, ενώ στη συνέχεια δεν απαιτείται κάποια συνδρομή. Τέλος, η διαδικασία συντηρησης της εφαρμογής από τους σχεδιαστές μας είναι επί πληρωμή. \newline \par
Λαμβάνοντας υπόψιν τα παραπάνω η εφαρμογή είναι οικονομικά βιώσιμη χωρίς κάποια χρηματοδότητση.

\subsection{Τεχνική Σκοπιμότητα}

Για την υλοποίση του project θα γίνει χρήση της γλώσσας προγραμματισμού Python, ενώ θα χρησιμοποιηθεί και το εργαλείο Figma για τη δημιουργία mockup-screens. Τέλος, για τη δημιουργία του logo της εφαρμογής χρησιμοποιήθηκε η ιστοσελίδα "freelogodesign.org". Τα παραπάνω εργαλεία είναι δωρεάν διαθέσιμα ενώ για τη χρήση του καθενός υπάρχουν μέλη στην ομάδα που διαθέτουν εμπειρία πάνω σε αυτά. Σύμφωνα με τα παραπάνω η εφαρμογή είναι υλοποιήσιμη από τεχνικής άποψης.

\subsection{Σκοπιμότητα Πόρων και Χρόνου}

Οι πόροι που απαιτεί η υλοποίηση αυτή περιλαμβάνουν κάποια συσκευή προγραμματισμού (προσωπικοί υπολογιστές), χώρος υποδοχής/εξυπηρετητής (αρχικά δωρεάν διαθέσιμος), προγραμματιστικά εργαλεία (δωρεάν διαθέσιμα), προγραμματιστές. \newline \par
Η ομάδα θέτει ως προτεραιότητα την υλοποίηση αυτού του project. Κάθε μέλος αναλαμβάνει να φέρει εις πέρας ορισμένα tasks κάθε φορά με βάση τις δυνατότητες του κάθε μέλους. Παράλληλα, γίνονται συχνές συναντήσεις για να ενημερώνονται όλα τα μέλη για την πρόοδο των υπολοίπων, ενώ ακόμη μέσω του Github δημοσιεύονται και
σε ηλεκτρονική μορφή τα αρχεία που δημιουργεί κάθε μέλος, ώστε να έχουν όλοι προσβαση σε αυτά. \newline \par
Με τα παραπάνω δεδομένα η εφαρμογή κρίνεται υλοποιήσιμη από άποψη χρόνου και πόρων.

\subsection{Λειτουργική Σκοπιμότητα}

Εφόσον δύο από τους σχεδιαστές μας έχουν εμπειρία στο UX design, η εφαρμογή θα σχεδιαστεί με βάση τους κανόνες του ανθρωποκεντρικού σχεδιασμού με σκοπό τη μέγιστη δυνατή χρηστικότητα της. Κατανοούμε ότι πρόκειται για ανθρώπους πολυάσχολους, όπου συνήθως το παραμικρό δευτερόλεπτο διαδραματίζει σπουδαίο ρόλο στην εξέλιξη της κατάστασης ενός ασθενούς, οπότε στόχος μας είναι ένα απλό design όπου σε συνδυασμό με τις λειτουργίες που θα προσφέρει η εφαρμογή να αποτελέσει αρωγή στο έργο των επαγγελματιών της υγείας. Λαμβάνοντας υπόψιν τα παραπάνω, κρίνεται ότι η εφαρμογή είναι υλοποιήσιμη από λειτουργικής άποψης.

\subsection{Νομική Σκοπιμότητα}

Από νομικής σκοπιάς η εφαρμογή καταγράφει προσωπικά ιατρικά δεδομένα ασθενών, οπότε η προστασία των προσωπικών δεδομένων των φυσικών προσώπων που θα καταγραφούν στις καρτέλες αποτελεί ένα σημαντικό ζήτημα. Παρόλα αυτά, εφόσον όλα τα στοιχεία αποθηκεύονται στα αρχεία των νοσοκομειακών ιδρυμάτων δεν αποτελεί ευθύνη των σχεδιαστών η κακομεταχείριση των δεδομένων αυτών. Οποιαδήποτε διαρροή προσωπικών στοιχείων των ασθενών ή ιατρικού προσωπικού σε τρίτους αποτελεί ευθύνη του νοσοκομειακού ιδρύματος και μόνο. \newline \par
\textbf{Σημειώνεται ότι η αγορά και η εγκατάσταση της εφαρμογής θα πραγματοποιείται επειτά από σειρά συζητήσεων με το εκάστοτε ίδρυμα, ώστε οι υπεύθυνοι να είναι βέβαιοι για γνησιότητα των αγοραστών.}

\end{document}

