\documentclass{article}

\makeatletter
\renewcommand{\fnum@figure}{Εικόνα \thefigure}
\makeatother

\usepackage[greek, english]{babel}
\usepackage{alphabeta}
\usepackage{atbegshi, picture}

% Set page size and margins
% Replace letterpaper' witha4paper' for UK/EU standard size
\usepackage[letterpaper,top=2cm,bottom=2cm,left=3cm,right=3cm,marginparwidth=1.75cm]{geometry}
\newcommand\T{\rule{0pt}{2.6ex}}       % Top strut
\newcommand\B{\rule[-1.2ex]{0pt}{0pt}} 
% Useful packages
\usepackage{amsmath}
\usepackage{graphicx}
\usepackage[colorlinks=true, allcolors=blue]{hyperref}
\usepackage[utf8]{inputenc}
\usepackage{indentfirst}
\usepackage{hyperref} 
 \hypersetup{ 
     colorlinks=true, 
     linkcolor=blue, 
     filecolor=blue, 
     citecolor = black,       
     urlcolor=black, 
     } 

\addto\captionsenglish{
  \renewcommand{\contentsname}
    {Περιεχόμενα}
}

% \title{Feasibility Study}
% \date{}

\begin{document}
% \maketitle

\begin{titlepage}
   \begin{center}
       \vspace*{1cm}

       \textbf{\huge Project Plan}

       \vspace{0.5cm}
        Τεχνολογία Λογισμικού
            
       \vspace{1cm}

       \textbf{Βλαχογιάννης Δημήτριος\\ Κούρου Αγγελική}
       
       
       \begin{figure}[!htb]
        \centering
        \includegraphics[width=0.5\textwidth]{905e4125666e44b594d9bd286a1d1b61.png}
        \end{figure}
        
        \vspace{0.5cm}
        
        \begin{figure}[!htb]
        \centering
        \includegraphics[width=0.5\textwidth]{up_2017_logo_en.jpg}
        \end{figure}


       \vfill
            
       Τεχνικό Κείμενο για την Τεχνολογία Λογισμικού\\
            
       \vspace{0.5cm}
            
       CEID, ECE\\
       University of Patras\\
            
   \end{center}
\end{titlepage}



\noindent Η ομάδα μας

\begin{enumerate}
  \item Βεργίνης Δημήτριος, ΑΜ: 10166634 , ECE
  \item Βλαχογιάννης Δημήτριος, ΑΜ: 1067371, CEID
  \item Κούρου Αγγελική, ΑΜ: 1067499 , CEID
  \item Μητροπούλου Αικατερίνα - Quality Manager, ΑΜ: 1067409, CEID
  \item Στεφανίδης Μάριος - Project Manager, ΑΜ:1067458, CEID
\end{enumerate}

{
  \hypersetup{linkcolor=black}
  \tableofcontents
}

\section{Εισαγωγή}
Κάνουμε την παραδοχή ότι η ομάδα μας αποτελείται από άτομα που έχουν ήδη μια εργασιακή απασχόληση στο χώρο της ανάπτυξης λογισμικού και τους έχει ανατεθεί η υλοποίηση μίας εφαρμογής για την βελτίωση της οργάνωσης ενός νοσοκομειακόυ περιβάλλοντος.\\ \\
\textbf{Σημείωση:} Όλο το κείμενο που αναπτύσσεται παρακάτω θα συνταχθεί από την οπτική μιας τέτοιας ομάδας.\\

\subsection{Ομάδα Υλοποίησης}
Έχοντας μόλις αποφοιτήσει η ομάδα προσεγγίζεται από μια νοσοκομειακή μονάδα για την ανάπτυξη ενός λογισμικού για τον πιο εύρυθμο συντονισμό του. Από την ομάδα επιλέχθηκε ομόφωνα για το ρόλο του Project Manager ο Στεφανίδης Μάριος.\\

\textbf{Σημείωση:} Οι παρακάτω ρόλοι δεν αντιστοιχούν στην πραγματική τους έννοια αλλά δίνονται με βάση τις γνώσεις και την εμπειρία του κάθε ατόμου στον τομέα του.
\begin{enumerate}
    \item Βεργίνης Δημήτριος: \emph{Senior Developer}
    \item Βλαχογιάννης Δημήτριος: \emph{Senior Developer}
    \item Κούρου Αγγελική: \emph{Middle Developer}
    \item Μητροπούλου Αικατέρινα: \emph{UX Designer, Middle Developer}
    \item Στεφανίδης Μάριος: \emph{UX Designer, Senior Developer}
\end{enumerate}

\subsection{Εργαλεία Υλοποίησης}

\begin{itemize}
    \item Γλώσσες Προγραμματισμού: Python, SQL
    \item Περιβάλλοντα Ανάπτυξης: PyCharm, Visual Studio Code, MariaDB
    \item Deployment Platform: Docker
    \item Πλατφόρμες Επικοινωνίας: Discord, Github (Issues,Commits), e-mail
    \item Διαγράμματα Gantt: \underline{\href{www.monday.com}{Monday}}
    \item Διαγράμματα Pert, ER : \underline{\href{www.lucid.app}{Lucidchart}}
    \item Επεξεργασία Τεχνικών Κειμένων: \underline{\href{www.overfleaf.com}{Overleaf}}
    \item Mock-up Screens: \underline{\href{www.figma.com}{Figma}}
    \item Version Control: \underline{\href{www.github.com}{Github}}
    \item Database Design: \underline{\href{www.dbdesigner.net}{DBDesigner}}
    % Ίσως μπει trello για Kanban
\end{itemize}

\section{Ανάθεση Εργασιών}
Οι εργασίες θα κατανεμηθούν με βάση τα σημεία που το κάθε μέλος της ομάδας υπερτερεί. Ύστερα από συνάντηση του Project Team οι αρμοδιότητα χωρίζονται ως εξής:

\begin{itemize}
    \item UX/UI Design: Μητροπούλου Αικατερίνα, Στεφανίδης Μάριος
    \item Front-end Developers: Κούρου Αγγελική, Μητροπούλου Αικατερίνα, Στεφανίδης Μάριος
    \item Back-end Developers: Βεργίνης Δημήτριος, Βλαχογιάννης Δημήτριος
    \item Database Design: Βεργίνης Δημήτριος, Βλαχογιάννης Δημήτριος
    \item Υπεύθυνος Χρονοπρογραμματισμού: Στεφανίδης Μάριος
    \item Alpha Testing: Βλαχογιάννης Δημήτριος, Κούρου Αγγελική, Μητροπούλου Αικατερίνα
    \item Quality Assurance: Μητροπούλου Αικατερίνα, Στεφανίδης Μάριος
    \item Υπεύθυνοι Επικοινωνίας (με τον πελάτη): Βεργίνης Δημήτριος, Κούρου Αγγελική 
\end{itemize}

\section{Κόστος Υλοποίησης}
Οι υπολογισμοί για το κόστος θα γίνουν με γνώμονα ότι το έργο θα έχει υλοποιηθεί στα επόμενα τρία χρόνια.

\subsection{Άμεσο Κόστος}
Μηνιαίες αποδοχές ομάδας:

 \begin{center}
     \begin{tabular}{|l|l|}
     \hline
      \textbf{Μέλος}   & \textbf{Ποσό} \T\B \\ 
      \hline
      Στεφανίδης Μάριος & 950\texteuro \T\B \\
      \hline
      Μητροπούλου Αικατερίνα& 850\texteuro \T\B \\
      \hline
      Βεργίνης Δημήτριος & 800\texteuro \T\B \\
      \hline
      Κούρου Αγγελική & 800\texteuro \T\B \\
      \hline
      Βλαχογιάννης Δημήτριος & 800\texteuro \T\B \\
      \hline
     \end{tabular}
 \end{center}
 \vspace{0.3cm}
Με βάση τον παρακάνω πίνακα προκύπτει ένα κόστος τριετίας {152.280,00\texteuro} για τη μισθοδοσία της ομάδας.

\subsection{Έμμεσο Κόστος}
 \begin{center}
     \begin{tabular}{|l|l|}
     \hline
      \textbf{Εφαρμογή}   & \textbf{Ποσό/Μήνα} \T\B \\ 
      \hline
      PyCharm & 21\texteuro \T\B \\
      \hline
      Monday & 45\texteuro \T\B \\
      \hline
      Lucidchart & 22\texteuro \T\B \\
      \hline
      Figma & 22\texteuro \T\B \\
      \hline
      Github Pro & 180\texteuro \T\B \\
      \hline
      DBDesigner & 30\texteuro \T\B \\
      \hline
     \end{tabular}
 \end{center}

\vspace{0.3cm}
     
Τα παραπάνω προγράμματα μαζί με τα κόστη τους αποτελούν τα επαγγελματικά εργαλεία που θα χρησιμοποιήσει η ομάδα μας προκειμένου να  αναπτύξει το \textbf{Medic World}. Ως ομάδα δεν απαιτούμε από τον πελάτη να εξασφαλίσει πλήρως τα κόστη αυτών των προγραμμάτων αλλά ένα ποσοστό της τάξεως του 15\%, το οποίο είναι σε βάθος τριετίας {1730\texteuro}.

\section{Διαγράμματα Χρονοπρογραμματισμού}

\subsection{Διαγράμματα Gantt}

Παρακάτω παρουσιάζονται τα διαγράμματα χρονοπρογραμματισμού που αφορούν την εξέλιξη του \textbf{Medic World}.

\vspace{0.3cm}

\begin{figure}[!htb]
\centering
\includegraphics[width=1\textwidth]{Gantt_Project_Plan_1.png}
\caption{\label{fig:Pert4} 01/03/2022 - 07/10/2023}
\end{figure}

\vspace{0.3cm}

\begin{figure}[!htb]
\centering
\includegraphics[width=1\textwidth]{Gant_Project_Plan_2.png}
\caption{\label{fig:Pert4} 08/10/2023 - 11/03/2025}
\end{figure}
\vspace{0.5cm}

\subsection{Διαγράμματα Pert}

Παρακάτω παρουσιάζονται τα Pert flowcharts που δημιουργήθηκαν με σκοπό την καλύτερη οργάνωση και συντονισμό των υποέργων/tasks του \textbf{Medic World}. Για περαιτέρω βοήθεια στην κατανόηση πρέπει να γίνουν οι εξής διευκρινίσεις:

\begin{itemize}
  \item Με κόκκινο βέλος συμβολίζεται το κρίσιμο μονοπάτι του έργου, οποιαδήποτε καθυστέρηση του οποίου θα οδηγήσει σε καθυστέρηση όλου του έργου
  \item Οι κίτρινοι ρόμβοι συμβολίζουν τα milestones του έργου.
\end{itemize}

\subsection{1$^o$ Milestone}

Επικοινωνία με τον πελάτη, παρουσίαση του prototype, συζήτηση για έγκριση και συνέχεια.

\vspace{0.3cm}

\begin{figure}[!htb]
\centering
\includegraphics[width=1\textwidth]{Pert_1I.png}
\caption{\label{fig:Pert1} 1$^o$ Milestone}
\end{figure}

\newpage

\begin{figure}[!htb]
\centering
\includegraphics[width=1\textwidth]{Pert_1II.png}
\caption{\label{fig:Pert2} 1$^o$ Milestone}
\end{figure}

\subsection{2$^o$ Milestone}

Πρώτη έκδοση του \textbf{Medic World} και παρουσίαση στον πελάτη.

\vspace{0.3cm}

\begin{figure}[!htb]
\centering
\includegraphics[width=1\textwidth]{Pert_2.png}
\caption{\label{fig:Pert3} 2$^o$ Milestone}
\end{figure}

\newpage

\subsection{3$^o$ Milestone}

Τελική παράδοση και έκδοση του \textbf{Medic World} μετά την διαδικασιά ελέγχου.

\vspace{0.3cm}

\begin{figure}[!htb]
\centering
\includegraphics[width=1\textwidth]{Pert_3.png}
\caption{\label{fig:Pert4} 3$^o$ Milestone}
\end{figure}

\vspace{0.5cm}

\textbf{Σημείωση:} Επικοινωνία με τον πελάτη θα διεξάγεται καθόλη την διάρκεια εκπόνησης του έργου και όχι μόνο κατά τις ημερομηνίες των milestones.

\end{document}