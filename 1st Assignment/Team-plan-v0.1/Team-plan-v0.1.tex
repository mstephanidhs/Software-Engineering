\documentclass{article}
\makeatletter
\renewcommand{\fnum@figure}{Εικόνα \thefigure}
\makeatother
\usepackage[greek, english]{babel}
\usepackage{alphabeta}
\usepackage{atbegshi, picture}
\usepackage[letterpaper,top=2cm,bottom=2cm,left=3cm,right=3cm,marginparwidth=1.75cm]{geometry}
\usepackage{amsmath}
\usepackage{graphicx}
\usepackage[colorlinks=true, allcolors=blue]{hyperref}
\usepackage[utf8]{inputenc}
\usepackage{indentfirst}
\usepackage[table]{xcolor}
\usepackage{hyperref} 
 \hypersetup{ 
     colorlinks=true, 
     linkcolor=blue, 
     filecolor=blue, 
     citecolor = black,       
     urlcolor=black, 
     } 

\addto\captionsenglish{
  \renewcommand{\contentsname}
    {Περιεχόμενα}
}

\begin{document}

\begin{titlepage}
   \begin{center}
       \vspace*{1cm}

       \textbf{\huge Team Plan}

       \vspace{0.5cm}
        Τεχνολογία Λογισμικού
            
       \vspace{1cm}

       \textbf{Κούρου Αγγελική\\Βλαχογιάννης Δημήτρης}
       
       \begin{figure}[!htb]
        \centering
        \includegraphics[width=0.5\textwidth]{905e4125666e44b594d9bd286a1d1b61.png}
        \end{figure}
        
        \vspace{0.5cm}
        
        \begin{figure}[!htb]
        \centering
        \includegraphics[width=0.5\textwidth]{up_2017_logo_en.jpg}
        \end{figure}


       \vfill
            
       Τεχνικό Κείμενο για την Τεχνολογία Λογισμικού\\
            
       \vspace{0.5cm}
            
        CEID, ECE \\
       University of Patras\\
            
   \end{center}
\end{titlepage}



\noindent Η ομάδα μας

\begin{enumerate}
  \item Βεργίνης Δημήτριος, ΑΜ: 10166634 , ECE
  \item Βλαχογιάννης Δημήτριος, ΑΜ: 1067371, CEID
  \item Κούρου Αγγελική, ΑΜ: 1067499 , CEID
  \item Μητροπούλου Αικατερίνα - Quality Manager, ΑΜ: 1067409, CEID
  \item Στεφανίδης Μάριος - Project Manager, ΑΜ:1067458, CEID
\end{enumerate}

{
  \hypersetup{linkcolor=black}
  \tableofcontents
}

\section{Εισαγωγή}
   
Στο παρόν τεχνικό κείμενο παρουσιάζεται ο τρόπος χειρισμού του έργου \textbf{Medic World} στο πλαίσιο του μαθήματος "Τεχνολογία Λογισμικού". Αρχικά θα δοθεί μια σύντομη επισκόπηση της ομάδας και στη συνέχεια θα αναλυθούν οι τρόποι και τα εργαλεία υλοποίησης της εργασίας καθώς και θα δοθούν τα διαγράμματα Gannt και Pert που αφορούν τον χρονοπρογραμματισμό της. 
 
 
\subsection{Ομάδα Υλοποίησης}

Για την σύσταση της ομάδας επιλέχθηκαν τα παραπάνω πέντε άτομα λαμβάνοντας υπόψιν τις δυνατότητες του καθενός και με ποιον τρόπο αυτές μπορούν να συνεισφέρουν σε μια ομάλη συνεργασία και στην ολοκλήρωση του έργου μας. Ξεκινώντας με τον Project Manager, Στεφανίδη Μάριο, ο οποίος με την οργανωτικότητα, τη σχολαστικότητα και τη συνέπειά του βοηθάει στον καλύτερο συντονισμό και την αποτελεσματική λειτουργία της ομάδας. Ως Quality Manager ορίστηκε η Μητροπούλου Αικατερίνα , η οποία ούσα ένα εργατικό, υπομονετικό και επικοινωνιακό άτομο διαχειρίζεται την εικόνα των εργασιών με σεβασμό απέναντι στους δημιουργούς τους.
\newline \par

Όσον αφορά στα Hard Skills της ομάδας φροντίσαμε τα μέλη που την απαρτίζουν να προσφέρουν το καθένα και κάτι διαφορετικό. Ο Βεργίνης Δημήτριος, θεωρούμε ότι θα διαδραματίσει σημαντικό ρόλο στην ανάπτυξη του κώδικα του \textbf{Medic World} με τις γνώσεις που διαθέτει στη γλώσσα \emph{Python}, ενώ παράλληλα φέρνει μια νέα οπτική εξαιτίας των σπουδών του στο τΗΜΤΥ. Ο Βλαχογιάννης Δημήτριος, έχοντας πολύπλευρες γνώσεις στο κομμάτι του προγραμματισμού καθώς και εφευρετικότητα παρέχει γρήγορα λύσεις στα πιθανά προβλήματα που θα προκύψουν. Η Κούρου Αγγελική έχοντας ακολουθήσει μία κατεύθυνση υλικού στις σπουδές της, θα διαχειριστεί τον κώδικα έχοντας ως γνώμονα την καλύτερη δυνατή αξιοποίηση των πόρων των συστημάτων με απώτερο στόχο τη λειτουργία του \textbf{Medic World} στο μέγιστο των δυνατοτήτων του. Τέλος, ανεξάρτητα από τις προγραμματιστικές τους γνώσεις, ο Στεφανίδης Μάριος και η Μητροπούλου Αικατερίνα, είναι εξοικειωμένοι με τους κανόνες του UX Design.


\subsection{Εργαλεία Υλοποίησης}

Για την υλοποίηση των επιμέρους διεργασιών του \textbf{Medic World} χρησιμοποιήθηκε μια πληθώρα εργαλείων όπως φαίνεται παρακάτω.


\begin{itemize}
    \item Γλώσσες Προγραμματισμού: Python, SQL
    \item Περιβάλλοντα Ανάπτυξης: PyCharm, Visual Studio Code, MariaDB
    \item Πλατφόρμες Επικοινωνίας: Discord, Githhub (Issues \& Commits), e-mail
    \item Διαγράμματα Gantt: \underline{\href{https://www.monday.com}{Monday}}
    \item Διαγράμματα Pert,ER: \underline{\href{https://lucid.app}{Lucidchart}}
    \item Επεξεργασία Τεχνικών Κειμένων: \underline{\href{https://www.overleaf.com}{Overleaf}}
    \item Mock-up Screens: \underline{\href{https://www.figma.com}{Figma}}
    \item Version Control: \underline{\href{https://github.com/}{Github}}
    \item Database Design: \underline{\href{www.dbdesigner.net}{DBDesigner}}
\end{itemize}
 

\subsection{Μέθοδος Ανάπτυξης Λογισμικού}

Ύστερα από συγκεντρωτική ψηφοφορία, επιλέχθηκε η μέθοδος ανάπτυξης λογισμικού \textbf{Scrum} για την υλοποίηση αυτού του έργου. Θεωρήθηκε κατάλληλη η παραπάνω μέθοδος για διάφορους λόγους. Πρώτον, στα πλαίσια ενός εξαμήνου, για να ολοκληρωθούν 6 παραδοτέα θα χρειαστούν sprint cycles για την ολοκλήρωση του project. Στο τέλος κάθε κύκλου εργασίας, θα αξιολογείται η πρόοδος της ομάδας από τους "πελάτες" (διδάσκοντες). Δεύτερον, λόγω του μεγέθους της ομάδας, κρίνεται απαραίτητος ο ηγετικός ρόλος του \textbf{Scrum Master} ή \textbf{Project Manager}, για να οργανώσει τις επιμέρους εργασίες και τα μέλη. Ακόμα, το γεγονός πως η ομάδα παρουσιάζει ετερογένεια, όσον αφορά στις ακαδημαϊκές σπουδές και στις διαφορές των εβδομαδιαίων προγραμμάτων του κάθε μέλους, συνιστά έναν ακόμη λόγο να υπάρχει κάποιος υπεύθυνος που θα συνεννοείται με όλους τους developers, ώστε να ολοκληρώνονται με επιτυχία τα διάφορα tasks. Λαμβάνοντας, λοιπόν, υπόψιν τα παραπάνω, επιλέχθηκε το \textbf{Scrum} ως ιδανικός τρόπος ανάπτυξης λογισμικού.

\section{Διαγράμματα χρονοπρογραμματισμού}

Παρακάτω παρουσιάζονται τα διαγράμματα χρονοπρογραμματισμού Gantt και Pert Charts.

\subsection{Διαγράμματα Gannt}

Τα παρακάτω διαγράμματα χρονοπρογραμματισμού δημιουργήθηκαν σύμφωνα με τις οδηγίες και τα χρονικά περιθώρια που δόθηκαν από τους διδάσκοντες του μαθήματος.

\vspace{0.3cm}

\begin{figure}[!htb]
\centering
\includegraphics[width=0.9\textwidth]{Gantt1.png}
\caption{\label{fig:Gantt1} Παραδοτέο 1$^o$}
\end{figure}

\newpage
 
\begin{figure}[!htb]
\centering
\includegraphics[width=1.0\textwidth]{Gantt2.png}
\caption{\label{fig:Gantt2} Παραδοτέo 2$^o$}
\end{figure}

\begin{figure}[!htb]
\centering
\includegraphics[width=1.0\textwidth]{Gantt3.png}
\caption{\label{fig:Gannt3} Παραδοτέo 3$^o$}
\end{figure}

\begin{figure}[!htb]
\centering
\includegraphics[width=1.0\textwidth]{Gantt4.png}
\caption{\label{fig:Gannt4} Παραδοτέo 4$^o$}
\end{figure}

\newpage

\begin{figure}[!htb]
\centering
\includegraphics[width=1.0\textwidth]{Gantt5.png}
\caption{\label{fig:Gannt5} Παραδοτέo 5$^o$}
\end{figure}

\begin{figure}[!htb]
\centering
\includegraphics[width=1.0\textwidth]{Gantt6.png}
\caption{\label{fig:Gannt6} Παραδοτέo 6$^o$}
\end{figure}


\subsection{Διαγράμματα Pert}

Τα Pert flowcharts που αφορούν τα παραδοτέα της εργασίας παρουσιάζονται παρακάτω. Για περαιτέρω βοήθεια στην κατανόηση πρέπει να γίνουν οι εξής διευκρινίσεις:

\begin{itemize}
  \item Με κόκκινο βέλος συμβολίζεται το κρίσιμο μονοπάτι του έργου, οποιαδήποτε καθυστέρηση του οποίου θα οδηγήσει σε καθυστέρηση όλου του έργου
  \item Οι κίτρινοι ρόμβοι συμβολίζουν τα milestones, επομένως στην προκειμένη περίπτωση τις ολοκληρωμένες παραδόσεις της εργασίας.
\end{itemize}

\begin{figure}[!htb]
\centering
\includegraphics[width=1\textwidth]{Pert1.png}
\caption{\label{fig:Pert1} Παραδοτέο 1$^o$}
\end{figure}

\newpage

\begin{figure}[!htb]
\centering
\includegraphics[width=1\textwidth]{Pert2.png}
\caption{\label{fig:Pert2} Παραδοτέο 2$^o$}
\end{figure}

\begin{figure}[!htb]
\centering
\includegraphics[width=1\textwidth]{Pert3.png}
\caption{\label{fig:Pert3} Παραδοτέο 3$^o$}
\end{figure}

\begin{figure}[!htb]
\centering
\includegraphics[width=1\textwidth]{Pert4.png}
\caption{\label{fig:Pert4} Παραδοτέο 4$^o$}
\end{figure}

\newpage

\begin{figure}[!htb]
\centering
\includegraphics[width=1\textwidth]{Pert5.png}
\caption{\label{fig:Pert5} Παραδοτέο 5$^o$}
\end{figure}

\begin{figure}[!htb]
\centering
\includegraphics[width=1\textwidth]{Pert6.png}
\caption{\label{fig:Pert6} Παραδοτέο 6$^o$}
\end{figure}




\end{document}
