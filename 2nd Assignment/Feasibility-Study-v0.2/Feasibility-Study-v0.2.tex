\documentclass{article}

\makeatletter
\renewcommand{\fnum@figure}{Εικόνα \thefigure}
\makeatother

\usepackage[greek, english]{babel}
\usepackage{alphabeta}
\usepackage{atbegshi, picture}

% Set page size and margins
% Replace `letterpaper' with`a4paper' for UK/EU standard size
\usepackage[letterpaper,top=2cm,bottom=2cm,left=3cm,right=3cm,marginparwidth=1.75cm]{geometry}

% Useful packages
\usepackage{amsmath}
\usepackage{graphicx}
\usepackage[colorlinks=true, allcolors=blue]{hyperref}
\usepackage[utf8]{inputenc}
\usepackage{indentfirst}
\usepackage{hyperref} 
 \hypersetup{ 
     colorlinks=true, 
     linkcolor=blue, 
     filecolor=blue, 
     citecolor = black,       
     urlcolor=black, 
     } 

\addto\captionsenglish{
  \renewcommand{\contentsname}
    {Περιεχόμενα}
}

% \title{Feasibility Study}
% \date{}

\begin{document}
% \maketitle

\begin{titlepage}
   \begin{center}
       \vspace*{1cm}

       \textbf{\huge Feasibility Study}

       \vspace{0.5cm}
        Τεχνολογία Λογισμικού
            
       \vspace{1cm}

       \textbf{Κατερίνα Μητροπούλου}
       
       \begin{figure}[!htb]
        \centering
        \includegraphics[width=0.5\textwidth]{logo.png}
        \end{figure}
        
        \vspace{0.5cm}
        
        \begin{figure}[!htb]
        \centering
        \includegraphics[width=0.5\textwidth]{UoP.jpg}
        \end{figure}


       \vfill
            
       Τεχνικό Κείμενο για την Τεχνολογία Λογισμικού\\
            
       \vspace{0.5cm}
            
       CEID, ECE\\
       University of Patras\\
            
   \end{center}
\end{titlepage}



\noindent Η ομάδα μας

\begin{enumerate}
  \item Βεργίνης Δημήτριος, ΑΜ: 1066634 , ECE
  \item Βλαχογιάννης Δημήτριος, ΑΜ: 1067371, CEID
  \item Κούρου Αγγελική, ΑΜ: 1067499 , CEID
  \item Μητροπούλου Αικατερίνα - Quality Manager, ΑΜ: 1067409, CEID
  \item Στεφανίδης Μάριος - Project Manager, ΑΜ:1067458, CEID
\end{enumerate}

{
  \hypersetup{linkcolor=black}
  \tableofcontents
}

\section{Ανάλυση Σκοπιμότητας}

Στο συγκεκριμένο τεχνικό κείμενο θα παρουσιαστεί μία εκτενής έρευνα που διεξάγεται γύρω από όλους τους παράγοντες που θα επηρεάσουν την υλοποίηση του project, ώστε να εκτιμηθεί το κατά πόσο αυτό είναι υλοποιήσιμο.

\subsection{Οικονομική Σκοπιμότητα}

Από τη στιγμή που πρόκειται για μία εφαρμογή που αποθηκεύει δεδομένα (για τους αθενείς και το ιατρικό προσωπικό) είναι απαραίτητη η χρήση ενός εξυπηρετητή (server). Η αποθήκευση όμως των δεδομένων θα γίνεται στους server του εκάστοτε νοσοκομείου, επομένως δεν απαιτείται η ενοικίαση ή η αγορά κάποιου server από την ομάδα σχεδίασης της εφαρμογής, η ευθύνη όμως για την αγορά (σε περίπτωση που δεν υπάρχει ήδη) η ενοικίαση ενός server μετατίθεται στον πελάτη-ίδρυμα.  \newline \par

Στο συνολικό προϋπολογισμό προστίθεται ένα μικρό ποσοστό των συνδρομών που αφορούν τα επαγγελματικά πακέτα/πλάνα των εργαλείων που θα χρησιμοποιηθούν κατά την διάρκεια σχεδιασμού και υλοποίησης του έργου ως ομάδα ανάπτυξης λογισμικού. \newline\par

Για να έχουν πρόσβαση τα νοσοκομειακά ιδρύματα στην εφαρμογή θα χρειαστεί να την αγοράσουν, ενώ στη συνέχεια δεν απαιτείται κάποια συνδρομή. Τέλος, η διαδικασία συντήρησης της εφαρμογής από τους σχεδιαστές μας είναι επί πληρωμή. \newline \par

Λαμβάνοντας υπόψιν τα παραπάνω η Medic World και στο πλαίσιο υλοποίησης της (νοσοκομείο), κρίνεται οικονομικά βιώσιμη. \newline \par

\textbf{Σημείωση:} Κόστος ενοικίασης/αγοράς server: 5 έως 40 δολάρια ανά μήνα αν πρόκειται για cloud server, ενώ αν πρόκειται για κανονικό server υπολογίζεται στα 100 με 200 δολάρια ανά μήνα. Τελευταία επιλογή είναι η αγορά ενός server που αντιστοιχεί στο χρηματικό ποσό των 1.000 με 3.000 δολαρίων. Λόγω της φύσης της επιχείρησης κρίνεται απαραίτητη η αγορά ενός server και όχι απλά η ενοικίαση.

\subsection{Τεχνική Σκοπιμότητα}

Για την υλοποίση του project θα γίνει χρήση της γλώσσας προγραμματισμού Python ενώ το περιβάλλον (IDE) που θα χρησιμοποιηθεί είναι το \emph{PyCharm} καθώς και το \emph{Visual Studio Code}. Για την αποθήκευση των απαραίτητων δεδομένων που απαιτεί η εφαρμογή, θα χρησιμοποιηθεί το εργαλείο σχεδιασμού βάσεων δεδομένων, \emph{MariaDB}. Το prototyping tool που επιλέχθηκε για την δημιουργία των mock-up screens 
είναι το \emph{Figma}, το οποίο με την interactive λειτουργεία που διαθέτει, έδωσε μια πρώτη ιδεά για τη μορφή και τη χρηστικότητα της εφαρμογής. \newline \par

Κάποιες άλλες από τις εφαρμογές που χρησιμοποιήθηκαν είναι \underline{\href{www.tablesgenerator.com}{Table Generator}} και η \underline{\href{www.freelogodesign.org}{FreeLogo Design}}. \newline\par
Σχετικά με την ασφάλεια της εφαρμογής, παρόλο που ο διακομιστής θα είναι τοπικός, η ομάδα αναλαμβάνει την ευθύνη να σχεδιάσει ένα λογισμικό το οποίο δεν είναι επιρρεπές σε επιθέσεις (π.χ. SQL injection attacks).
\newline \par
Το κάθε εργαλείο έχει χρησιμοποιηθεί σε ικανοποιητικό βαθμό στο παρελθόν από όλα τα μέλη της ομάδας, συνεπώς για κάθε μέρος του έργου υπάρχει άτομο με την κατάλληλη εμπειρία και γνώσεις πάνω στο αντικείμενο. 
\newline\par

Σύμφωνα με την τεχνογωνσία του κάθε μέλους της ομάδας η υλοποίηση της Medic World από τεχνικής άποψης κρίνεται εφικτή.

\subsection{Σκοπιμότητα Πόρων και Χρόνου}

Για την υλοποίηση του έργου απαιτούνται  πόροι, όπως προσωπικοί υπολογιστές για τον προγραμματισμό της εφαρμογής, πρόσβαση σε νοσοκομειακά μηχανήματα (monitors) για τον έλεγχο της λειτουργίας της εφαρμογής σχετικά με το κομμάτι που αφορά την παρακολούθηση των ζωτικών ενδείξεων των ασθενών, χώρος υποδοχής/εξυπηρετητής (από τον πελάτη), προγραμματιστικά εργαλεία και προγραμματιστές. \newline \par

Η ομάδα θέτει ως προτεραιότητα την υλοποίηση αυτού του project. Κάθε μέλος αναλαμβάνει να φέρει εις πέρας ορισμένα tasks κάθε φορά με βάση τις δυνατότητες του. Παράλληλα, γίνονται συχνές συναντήσεις για να ενημερώνονται όλα τα μέλη για την πρόοδο των υπολοίπων, ενώ ακόμη μέσω του Github δημοσιεύονται και σε ηλεκτρονική μορφή τα αρχεία που δημιουργεί κάθε μέλος, ώστε να έχουν όλοι προσβαση σε αυτά. \newline \par

Με τα παραπάνω δεδομένα χωρίς να υπολογίστούν απρόοπτες καθυστερήσεις ή άλλα γεγονότα που θα αποτελέσουν τροχοπέδη για την διεκπεραίωση του έργου, η ολοκλήρωση του κρίνεται εφικτή από άποψη χρόνου και πόρων.

\subsection{Λειτουργική Σκοπιμότητα}

Εφόσον δύο από τους σχεδιαστές μας έχουν εμπειρία στο UX Design, η εφαρμογή θα σχεδιαστεί με βάση τους κανόνες του ανθρωποκεντρικού σχεδιασμού με σκοπό τη μέγιστη δυνατή χρηστικότητα της. Κατανοούμε ότι πρόκειται για ανθρώπους πολυάσχολους, όπου συνήθως το παραμικρό δευτερόλεπτο διαδραματίζει σπουδαίο ρόλο στην εξέλιξη της κατάστασης ενός ασθενούς, συνεπώς στόχος μας είναι ένα απλό design όπου σε συνδυασμό με τις λειτουργίες που θα προσφέρει η εφαρμογή να αποτελέσει αρωγή στο έργο των επαγγελματιών της υγείας.\newline\par

Λαμβάνοντας υπόψιν τα παραπάνω, κρίνεται ότι η εφαρμογή θα εξελίξει το νοσοκομειακό μηχανισμό και το ήδη υπάρχον σύστημα διαχείρησης ασθενών, επομένως από λειτουργικής απόψης η Medic World κρίνεται εφικτή.

\newpage

\subsection{Νομική Σκοπιμότητα}

Έρευνα για τα νομικά πλαίσια που απασχολούν την ανάπτυξη και την υλοποίηση του \textbf{Medic World}. 

\subsubsection{Προσωπικα Δεδομένα Ασθενών & Προσωπικού}
Από νομικής άποψης η εφαρμογή καταγράφει προσωπικά ιατρικά δεδομένα ασθενών, οπότε η προστασία των προσωπικών δεδομένων των φυσικών προσώπων που θα καταγραφούν στις καρτέλες αποτελεί ένα σημαντικό ζήτημα. Παρόλα αυτά, εφόσον όλα τα στοιχεία αποθηκεύονται στα αρχεία των νοσοκομειακών ιδρυμάτων δεν αποτελεί ευθύνη των σχεδιαστών η κακόβουλη χρήση των δεδομένων αυτών. Οποιαδήποτε διαρροή προσωπικών στοιχείων των ασθενών ή ιατροφαρμακευτικού προσωπικού σε τρίτους αποτελεί ευθύνη του νοσοκομειακού ιδρύματος και μόνο. \newline \par


\textbf{Σημειώνεται ότι η αγορά και η εγκατάσταση της εφαρμογής θα πραγματοποιείται έπειτα από σειρά συζητήσεων του εκάστοτε ιδρύματος με το νομικό τμήμα, ώστε οι υπεύθυνοι να είναι βέβαιοι για γνησιότητα των αγοραστών αλλά και να ακολουθηθούν οι αρμόζουσες νομικές διαδικασίες που αφορούν το απόρρητο των προσωπικών δεδομένων.}


\subsubsection{Copyright Issues}

Εφόσον το design της εφαρμογής για το τμήμα που αφορά τα μέσα κοινωνικής δικτύωσης έχει στηριχθεί σε ήδη υπάρχουσες εφαρμογές, όπως το Facebook και το Messenger, η ομάδα σχεδίασης ως ένα βαθμό οφείλει να διαφοροποιήσει τη διεπαφή του \textbf{Medic World}, ώστε να αποφευχθεί ζήτημα κλοπής πνευματικών δικαιωμάτων, που προστατεύονται από το 2011 και μετά από τα guidelines της εταιρείας. 
\par Επίσης κρίνεται σημαντικό να αναφερθεί ότι οι δύο εφαρμογές απευθύνονται σε διαφορετικό κοινό, επομένως δε διατρέχει κίνδυνος αντικρουόμενων συμφερόντων, επομένως δεν πρόκειται να τεθεί ζήτημα μήνυσης για διαφυγόντα κέρδη από την εταιρεία (Meta). 
\par Συνεπώς η προστασία των πνευματικών δικαιωμάτων που καλύπτουν το interface των εφαρμογών, οι οποίες λειτούργησαν ως έμπενυση για το \textbf{Medic World}, κρίνεται ότι δε θα δημιουργήσουν κάποιο νομικό ζήτημα.

\subsubsection{Πολιτική Cookies}

Μία από τις λειτουργίες του \textbf{Medic World} είναι στην καρτέλα  News Article του Newsroom το σύστημα να προτείνει άρθρα στους χρήστες με βάση την επαγγελματική τους ιδιότητα και προηγούμενες αναζητήσεις τους, ώστε να προσφέρει μία πιο εξατομικευμένη εμπειρία. Για τον προαναφερθέντα λόγο κρίνεται απαραίτητη η ενημέρωση των χρηστών για τη συγκεκριμένη λειτουργία και η αίτηση για τη συνέναισή τους. 
\par Σημειώνεται ότι σε περίπτωση που ο χρήστης δεν αποδεχτεί τους όρους χρήσης, τότε η λειτουργία αυτή θα απενεργοποιείται για το συγκεκριμένο λογαριασμό. Για ρύθμιση των cookies ο εκάστοτε χρήστης θα μπορεί να ανατρέξει στις ρυθμίσεις της εφαρμογής ανά πάσα στιγμή.


\end{document}
